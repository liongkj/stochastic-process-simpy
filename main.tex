
\documentclass[a4paper,11]{article}

%\title{Unused Title}
\usepackage{graphicx}
\usepackage{hyperref}
\usepackage{multirow}
\usepackage{multicol}
\usepackage{blindtext}
\usepackage[utf8]{inputenc}
\usepackage[english]{babel}
\usepackage[T1]{fontenc}

% Use helvet if uarial cannot be installed
%\usepackage{uarial}
\usepackage{fontspec}
\setmainfont{Times New Roman}
\renewcommand{\familydefault}{\sfdefault}
\usepackage{amssymb}
\usepackage{amsmath}
\usepackage{courier}
\usepackage{setspace}
\usepackage[table,svgnames]{xcolor}
\usepackage{fancyvrb} 
\usepackage{listings}
\usepackage{caption}
\usepackage{longtable}
\usepackage{relsize}
\usepackage{tfrupee}
\usepackage{rotating}
\usepackage{lipsum}
\usepackage{subcaption}
\usepackage{float}
\usepackage{aliascnt}
\usepackage{pst-node}
\usepackage{enumitem}

\usepackage[nottoc]{tocbibind}
\usepackage[
backend=biber,
style=numeric,
sorting=none,
giveninits=true]
{biblatex}
\DeclareNameAlias{author}{last-first}
\addbibresource{ref.bib}
%\usepackage{}
%[square,sort,comma,number]
%\bibliographystyle{humannat}
%\setcitestyle{super, open={[},close={]}

\makeatletter
\newcommand\footnoteref[1]{\protected@xdef\@thefnmark{\ref{#1}}\@footnotemark}
\makeatother
\usepackage{listings}
\usepackage{capt-of} %image caption
\captionsetup[figure]{name={Fig.},labelsep=period}


\usepackage{titlesec}
\titleformat{\section}
    {\normalfont\fontsize{13}{17}\bfseries}{\thesection.}{1em}{\MakeUppercase}
% \titleformat{name=\section,numberless}
%     {\normalfont\fontsize{12}{15}\bfseries}{}{}{\MakeUppercase}
\titleformat{\subsection}
    {\normalfont\fontsize{12}{17}}{\thesubsection.}{1em}{}
   
\providecommand{\keywords}[1]{\textbf{\textit{Keywords: }} #1} 
\newaliascnt{eqfloat}{equation}
\newfloat{eqfloat}{h}{eqflts}
\floatname{eqfloat}{Equation}

\newcommand*{\ORGeqfloat}{}
\let\ORGeqfloat\eqfloat
\def\eqfloat{%
	\let\ORIGINALcaption\caption
	\def\caption{%
		\addtocounter{equation}{-1}%
		\ORIGINALcaption
	}%
	\ORGeqfloat
}

\addto\captionsenglish{% Replace "english" with the language you use
	\renewcommand{\contentsname}%
	{List of Contents}%
}

\newcommand\tab[1][1cm]{\hspace*{#1}}

\definecolor{codegreen}{rgb}{0,0.6,0}
\definecolor{codegray}{rgb}{0.5,0.5,0.5}
\definecolor{codepurple}{rgb}{0.58,0,0.82}
\definecolor{backcolour}{rgb}{0.95,0.95,0.92}

\lstdefinestyle{mystyle}{
	backgroundcolor=\color{backcolour},   
	commentstyle=\color{codegreen},
	keywordstyle=\color{magenta},
	numberstyle=\tiny\color{codegray},
	stringstyle=\color{codepurple},
	basicstyle=\ttfamily\footnotesize,
	breakatwhitespace=false,         
	breaklines=true,                 
	captionpos=b,                    
	keepspaces=true,                 
	numbers=left,                    
	numbersep=5pt,                  
	showspaces=false,                
	showstringspaces=false,
	showtabs=false,                  
	tabsize=2,
	xleftmargin=0.5cm,
	xrightmargin=-0.8cm,
	frame=lr,
	%	framesep=-5pt,
	framerule=0pt
}

\lstset{style=mystyle}

\definecolor{Teal}{RGB}{0,128,128}
\definecolor{NewBlue1}{RGB}{4,100,226}
\definecolor{NiceBlue}{RGB}{63,104,132}
\definecolor{DarkRed}{RGB}{14,53,59}
\definecolor{NewBlue2}{RGB}{62,100,125}
\definecolor{NewBlue3}{RGB}{44,100,128}

\hypersetup{
	colorlinks,
	citecolor=NiceBlue,
	linkcolor=NewBlue1,
	urlcolor=Blue
	%	citebordercolor=Violet,
	%	filebordercolor=Red,
	%	linkbordercolor=Blue
} %hyperlink styling

\usepackage{geometry}
\linespread{1.25}
\usepackage[parfill]{parskip} % Avoid indentation

\geometry{
	a4paper,
	left=2.5cm,
	right=2.5cm,
	top=2.5cm,
	bottom=2.5cm,
}
\begin{document}
\pagenumbering{gobble}
\begin{center}
    {\large SHANGHAI JIAO TONG UNIVERSITY}
\end{center}
%	\maketitle
\vspace{6cm}

\begin{center}

    \Huge The analysis and simulation of queuing system in two fast food chain\\
    \vspace{.5cm}


\end{center}
\vspace{2.5cm}
\begin{center}
    \Large Liong Khai Jiet\\Advisor: Shi Jian Hong
\end{center}

\vspace{8cm}
\begin{center}
    {\large A final project report submitted in partial fulfillment of the \\course ICE-6502 }
\end{center}

\begin{center}
    {\large August 2020}
\end{center}

\newpage
\pagenumbering{Roman} %page numbering before main pages
\tableofcontents
\pagebreak
\newpage

\cleardoublepage\pagenumbering{arabic}


\textbf{Introduction, background to research, overview of research, aims and objectives, terms of reference, key research questions}

\begin{itemize}
	\item Want to study the queuing system in two fast food chain affect the efficiency
	\item Mc Donalds is using the ticketing system where the customer order food from two counter and get a ticket while the cook prepare the food. KFC use a first in first out method where the cashier gets the order from the customer, send it to the kitchen and then the customer pays only when they receive the food.




\end{itemize}

Your final Report should be no longer than 8,000 words. The word limit applies to the main text area only, as this is where your report is introduced and then carried forward. Your final word-count should not include: title page, synopsis, acknowledgements, list of contents, notation (if applicable), references, bibliography or appendix/appendices. Shorter Reports are allowed depending on the nature of the research (for example some mathematical and statistical projects are usually shorter than qualitative projects). Reports longer than 8,000 words will be penalised.\\
\\
If the Report is to contain a considerable number of tables and figures, it may be best to place them in an appendix and use in the main text only such summary tables or charts as will assist the reader in following your arguments without necessarily having to go into great detail. This may help to ensure a smooth and uninterrupted presentation.\\
\\
5.11 UNACCEPTABLE PRESENTATION
\textbf{Examiners will not accept projects where the presentational guidelines are not adhered to}. They will not accept projects that have any of the following:
Copyright statements (in any format including as a header/footer);
Confidentiality statements (in any format including as a header/footer);
Non standard font size, type and line spacing;
Page numbers missing;
No contents pages or contents pages with no page numbers indicated.


\section{Material and methods}
\subsection{Model}
The queuing model designed for the computer simulation describes the probabilistic nature of the two different queuing system. It allows mathematicians to use the constricted model to make predictions and study how an inefficient queuing system could affect customer experience (long waiting lines) and resource wastage (low utilization rate).
Most of the queuing problem could be classified into three parts, which are the input process, the service distribution and the queue discipline \cite{PINSKY2011447}.

A unlimited queuing simulation ($t_1,t_2,\dots,t_n$) will be carried out using a python package Simpy \cite{grayson} to simulate the restaurant operation hours.
The following assumptions applied in the simulations are listed here and discuss in the corresponding sections:
\begin{enumerate}[label=\alph*)]
    \item Infinite calling population
    \item First-come, first-served queue discipline
    \item Poisson arrival rate, $\lambda$
    \item Exponential service rate, $\mu_1,\mu_2$
    \item Customers only buys one set of food
    \item Customers does not return to the queue after completing a purchase.
    \item Kitchen cooks can only prepare one set of food a time.
    \item There is no waiting line between neighboring stages in the system with blocking.
\end{enumerate}
The following notation is used to represent the random variables associated with the model:
\begin{align*}
    N_q            & = \text{the number of customers in queue.}                             \\
    N_s            & = \text{the number of customers receiving service.}                    \\
    N              & = N_q +N_s = \text{the total number of customers in the system.}       \\\\
    X_s            & = \text{the time of a customer spends in taking order.}                \\
    X_k            & = \text{the time of a customer spends in waiting for food.}            \\
    X              & = X_s + X_k =  \text{the time of a customer spends in actual service.} \\\\
    T              & = W+X =\text{the total time a customer spend in the system.}           \\
    \lambda        & = \text{the arrival rate (average number of arrivals/min)}             \\
    \mu            & = \text{the service rate (average number served / min)}                \\
    \textbf{c}     & = \text{the number of counter services in parallel}                    \\
    \textbf{s}     & = \text{the number of cooks in the kitchen}                            \\
    \textbf{n}     & = \text{the number of customer served}                                 \\
    \textbf{S} & = \text{System}
\end{align*}

In Kendall's notation,

\subsubsection{Model of Customer Population}
The simulation also assumed the customer population in the model used in this supplement are  are infinite and patient, so they do not renege, balk or jockey to prevent the mathematical formula to become overly complex. Besides, the simulation will end when the the restaurant served 1000 customers ($N_{1000}$) the remaining customers will leave the system at once no matter which process they currently at. Therefore, the number of customer served will be only the ones who completed the whole process and the unfinished processes are discarded completely.
\subsubsection{Waiting Line Model}
In the experiment setting, the arrival rate of customers has a Poisson distribution with parameter $\lambda$, where each of them are allowed to only buy one set of food. Then, the customers will need to enter the waiting line with infinite N the customers can place order via \textbf{r} number of serving counters and the order is then sent to the kitchen to be prepared by \textbf{s} number of cooks. Each counter server and kitchen cook can only handle one task meal at a time, i.e. placing order and preparing food. Besides, the process of the customer placing order and kitchen preparing the meal are independent and has the exponential distribution with parameter $\mu$. The sequence of arrival process and placing order are all assumed to be mutually independent.

The system describes the characteristics of queue system, such as the number of waiting lines, the number of server and service patterns, etc. The two variables introduced in the experiment is the queue with blocking or without blocking, before the transition to the second stage of service point (food collection) which typically arises from the problem of lacking queuing capacity between jobs \cite{Gomez-Corral2002}.

Both model could be broken down into two stages where the first stage are identical in the sense that they have single-line, multiple server with First-come-first-serve discipline (M/M/c). The difference is multi-stage tandem-queuing model \cite{ross2014introduction} system showed in figure \ref{fig:single-phase}  and the latter  multi-stage wait-queuing model (G/M/∞) \cite{WU2019927} showed in figure \ref{fig:multi-phase}

\noindent
\begin{minipage}{\textwidth}
    \includegraphics[width=\textwidth]{images/queue_model_1.png}
    \captionof{figure}{Single-phase line model (w/ blocking)}\label{fig:single-phase}
\end{minipage}
\begin{minipage}{\textwidth}
    \includegraphics[width=\textwidth]{images/queue_model.png}
    \captionof{figure}{Multi-phase line model (w/o blocking)}\label{fig:multi-phase}
\end{minipage}
\subsubsection{Model of Service System}
In this project, we will study two types of queuing method, which are as follows:

\begin{enumerate}[label=\roman*)]
    \item $S_1$: Single line, Multi-server, Two-stage (w/ blocking) (\autoref{fig:single-phase}) \\
          Scenario: Customer arrival rate follows the Poisson process with the parameter $\lambda$. The arrived (n)th customer has to pass through two consecutive service point before leaving the system. The (n)th  customer will enter the only one waiting line with unlimited queue space at the first stage of serving point.
          Then, the customer will enter one of the free \textbf{r} counters and place an order but because there is no waiting space at the second stage of serving point. So the (n)th customer will need to wait at the counter while the kitchen is preparing the food.

          The kitchen will process the resources with  first-come, first-served priority rule. The server takes order from the customer with the service rate of exponential distribution parameter $\mu_1$.

          Next, the order is sent to the kitchen and the customer exits the system after they make the payment and receive the food.

    \item $S_2$: Single line, Multi-server, Two-stage (w/o blocking) (\autoref{fig:multi-phase}) \\
          Scenario: Customer arrival rate follows the Poisson process with the parameter $\lambda$; The arrived ($n$)th customer has to pass through two consecutive service point before leaving the system. The customers will enter the only one waiting line with unlimited queue space at the first stage of serving point.
          Then, the ($n$)th customer will enter one of the free \textbf{r} counters and place an order but because there is no waiting space at the second stage of serving point. So the customer will be blocked at the counter upon finish placing the order while the kitchen is preparing the food.

          The kitchen will process the resources with the first-come,first-served priority rule. The server takes order from the customer with the service rate of exponential distribution parameter $\mu_1$.

          After making the payment, the counter will send the order to the kitchen and starts to service the next customer whereas the customer proceeds to the collection area. The customer exits the system after they receive the food.
\end{enumerate}

\subsubsection{Model of Kitchen Process}
Cooks are the main resources in the kitchen. We model the \textit{\textbf{s}} cooks as fixed resources.
Once an item order ($i$)th arrives, it need to enter the order waiting queue, $\beta_{k}$ for a cook to be free.
Once the order is assigned to the cook, the cook will start preparing the food ($i$)th next in queue and cannot  prepare ($i+1$, $i+2$, \dots , $i_n$)th orders simultaneously. Thus, the food preparation time is modeled using the exponential distribution of parameter $\mu_2$.
Once complete, the cook will send the ($i$)th order back to the counter and the process is repeated for ($i+1$, $i+2$, \dots , $i_n$)th order.

\subsection{Performance measure}
The project focus on few metrics to analyze the queuing system model and obtain the estimation of their corresponding performance measure. Before that, we create the following project scope and performance metrics to help us understand and identify any bottleneck found in the two systems.

From the section above, we introduced that
\begin{align*}
    \lambda & =\text{ mean arrival rate} \\
    \mu     & = \text{mean service rate}
\end{align*}
Now, we use Little's Law to derive the formula below:
\begin{align*}
    \rho & = \frac{\lambda}{c\mu} = \text{System utilization rate}.                                  \\
    L_Q  & = \rho \dot L =\text{ Average number of customers waiting in line and its distribution.} \\
    W    & = \frac{1}{\mu - \lambda}=\text{Average time of a customer spends in the system.}        \\
    W_Q  & = \text{ Average time of a customer spend waiting.}
\end{align*}
The data findings allows us to understand the relationship between the number queuing method, average waiting time and system utilization percentage.
\medskip
\section{Results and Discussion}
In the simulation, the system with stage blocking are assigned to the name \textbf{FIFO} whereas the system without stage blocking are assigned with the name \textbf{TICKET}
\subsection{System utilization rate}
With $\lambda = 3$, $\mu = 1.5$ we plot the graph of how the number of servers affect the efficiency of each server. We can observe that \textbf{FIFO} system has lower utilization percentage (high idle times) because of the blocking mechanism at the second stage. The server resource is idle but it is blocked therefore it cannot process take order from another customer. In business context, this idling duration translates to business cost. Thus, the \textbf{TICKET} system with a waiting line in between the two stages can eventually maximize the resource utilization hence reaching the goal of 1000 customers served about 40\% faster compared to \textbf{FIFO}. 

The results are tabulated in table \ref{Tab:MainResults} with the parameters, $\lambda = 2, \mu_1 = 2 $ and $\mu_2 = 2$ 


	\begin{table}[ht]
		\caption{Simulations results with n = 1000}
		\begin{center}
 \begin{tabular}{|c|c|c|}
 	\hline
 	& \textbf{FIFO} & \textbf{TICKET}\\
 	\hline
 	Total Simulation Time & 909 & 513 \\
 	\hline
 	Average Queue Length & 223 & 5 \\
 	\hline
 	Max Queue Length (min) & 450 & 27 \\
 	\hline
 	Average Waiting Time (min) & 205 & 4.65 \\
 	\hline
 	Average Service Time (min) & 4.5 & 4.51 \\
 	\hline
 	Average Time Spent in System (min) & 207.33 & 8.51 \\
 	\hline
 \end{tabular}
\label{Tab:MainResults}
\end{center}
\end{table}

\noindent
\begin{minipage}{\textwidth}
    \includegraphics[width=\textwidth]{images/utilization_percentage.png}
    \captionof{figure}{$\rho$ = Relationship between $\lambda$ and system utilization percentage}\label{utilization_percentage}
\end{minipage}

\subsection{The average number of customers waiting in line}

The simulation results in left diagram in \ref{mean_queue_length} shows that average queue length of \textbf{FIFO} has the higher exponential rate of increase when arrival rate $\lambda$ increases compared to the \textbf{TICKET} system. The right graph also showed similar results when the service rate $\mu$ decreases, customers in \textbf{FIFO} system will have higher probability of facing longer queue. 

Indeed, the \textbf{FIFO} system with blocking is observed to be more easily affected by the increase of arrival rate (i.e. lunch hour) or decrease of service rate (staff shortage).

\noindent
\begin{minipage}{\textwidth}
    \includegraphics[width=\textwidth]{images/mean_queue_length.png}
    \captionof{figure}{$L_Q$ = Average queue length}\label{mean_queue_length}
\end{minipage}



\subsection{Average time of a customer spend in the system}

The average time of a customer spend in the system includes both queuing and service time. The simulation results shows similar conclusion where customers in  \textbf{FIFO} system has a higher probability of spending more time in the system.

\noindent
\begin{minipage}{\textwidth}
    \includegraphics[width=\textwidth]{images/mean_time_in_system.png}
    \captionof{figure}{$L_Q$ = Average time in system}\label{bar}
\end{minipage}
\subsection{Average time of a customer spend waiting}

The average time of a customer spend waiting includes both queuing and food collection process. The simulation results shows similar conclusion where customers in  \textbf{FIFO} system has a higher probability of spending more time waiting which is also observed in the simulation results compiled in table \ref{Tab:MainResults}.

\noindent
\begin{minipage}{\textwidth}
    \includegraphics[width=\textwidth]{images/mean_waiting.png}
    \captionof{figure}{$L_Q$ = Average time spend waiting}\label{bar}
\end{minipage}

\medskip
\section{\textsc{Conclusion}}
	In this study, we simulate the blocking problem in fast food chain and analyze the impacts of the service interaction and customer experience. We showed that the multi-tandem stage system has a correlation between the blocking and waiting times at the following stages for later customers. Based on personal experience and reviews from Google, the queue length and expected queue time increases significantly during lunch hours or when the restaurants offer limited merchandise in happy meals, long queue is  more common in KFC restaurants because of the efficiency of queuing system used compared to MCD. Although in practice, KFC servers most probably will ask the customer to wait aside while taking order for the next customer, simulation results still proved that MCD wait queuing system is more efficient, reduced wait times, have better staff utilization, customer flow management and minimizing operational cost.
\medskip
\section{\textsc{Acknowledgment}}
I am eternally grateful to my advisor, Prof. Shi Jian Hong for explaining about stochastic processes and queuing theory which provided me a good understanding and knowledge to complete this project. I would also like to thank my friend for her willingness to proof read my work and provide useful suggestions for my work. Finally, I would like to thank SJTU for allowing us to proceed our studies online in times of hardship like this and the canvas technical teams for the great support during our semester.

\pagebreak

\printbibliography
\pagebreak

\pagenumbering{roman}

\section{Appendix: Source code for simulation}
\lstset{language=Python}
\lstset{frame=lines}
\lstset{caption={fifo.py}}
\lstset{label={lst:codefifo}}
\lstset{basicstyle=\footnotesize}
\lstinputlisting[language=Python]{./code/fifo.py}

\lstset{caption={ticket.py}}
\lstset{label={lst:codeticket}}
\lstset{basicstyle=\footnotesize}
\lstinputlisting[language=Python]{./code/ticket.py}

% \lstset{caption={main.py}}
% \lstset{label={lst:codemain}}
% \lstset{basicstyle=\footnotesize}
% \lstinputlisting[language=Python]{./code/main.py}
 \end{document}
