
\begin{abstract}

	Fast food restaurants are popular among working adults and students who value the conducive environment and its convenient services. As such, fast food chain like McDonald's (MCD) and Kentucky Fried Chicken(KFC) are available in most places including shopping complex, office area and university cafeteria in Malaysia. Fast-food restaurants illustrate the transient nature of waiting line system, as they introduce promotions value meal time to time, resulting occasional long queues and inconvenient waiting times. For fast food, as the names states means it has a short service time. Thus, it is a suitable target for this project to analyse the performance measure of the multiple server single queue system of a restaurant.

\end{abstract}
\keywords{one, two, three, four}

\section{Introduction}
Queuing theory is the mathematical study of waiting lines which are often used to make predictions about how a system could cope with demands \cite{adan2002queueing}. In this project, we want to study the queuing system in two fast food chain affect the efficiency.
The queue times or total waiting times is the most significant part of a good customer experience\cite{ROY201629}, so the simulation focuses on analyzing how the two type of queuing system affects the average waiting times of the customer and the idle time of the serving counters.
Running a computer simulation is a efficient method to validate the analytical model of the system to pinpoint any bottleneck resources in the system. This also allows us to test the efficiency of solution approached by both fast food chain restaurant using stochastic and queuing theory.

The challenge here is to determine which service system has sufficient but not excessive amount of capacity which translates to cost in the business world. As for fast food restaurants, short queuing time and short waiting line is important in attracting more customers, putting aside the factors like good dining environment and delicious food\cite{dharmawirya2012analysis}. In this report, I will focus on the two largest fast food chain in Malaysia \cite{abdullah2015trend} and observe their queuing pattern. MC Donald's is using the multi-server, two-phase ticketing queue where the customer order food from two counter and get a ticket while the cook prepare the food\cite{Kohimprovemcd}. KFC use a multi-server, two-phase blocking queue where the cashier takes order from the customer, send it to the kitchen and then the payment is made when they receive the food. Although both are having single queue, they have different type of system which the former uses a number queue system for food collection and the latter uses a "blocking" style queuing system when customers transition to the next stage.

The objective of this project is to estimate how this blocking have impact upon the customer flow in a restaurant and its likelihood of a completely loaded system.
